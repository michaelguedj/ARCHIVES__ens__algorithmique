
%%%
\subsection{Définition du problème}
%%%
%%%
\blue{
\begin{fDefinition}[Cycle hamiltonien]
Un cycle hamiltonien (d'un graphe)
est un cycle passant par tous les sommets du graphe, une fois et une seule.
\end{fDefinition}
}

\blue{
\begin{fDefinition}[Problème du voyageur de commerce]
$~$
\begin{itemize} 
	\item \textup{Entrée.}
		\begin{itemize}
			\item[] Un graphe $G=(S, A, cost)$, tel que : 
			\begin{itemize}
				\item[--] $G$ est non orienté ; 
				\item[--] $G$ est complet ;
				\item[--] Les arcs de $G$ sont valués par la fonction $cost : A\rightarrow \mathbb{R}$.
			\end{itemize}
		\end{itemize}
	\item \textup{Problème.}
		\begin{itemize}
			\item[] Trouver le cycle hamiltonien ayant le coût le plus faible.
		\end{itemize}
\end{itemize}
\end{fDefinition}
}

\blue{
\begin{fTheorem}
Le problème du voyageur de commerce est NP-hard. 
\end{fTheorem}
}

\begin{proof}
Admis.
\end{proof}
%%%
%%%
%%%
%%%
%%%
%%%
\subsection{Heuristique : algorithme du plus proche voisin}
%%%
%%%

\begin{algorithm}[H]
\caption{$\algo{nearest\_neighbour}( G=(S, A, cost) )$}
\begin{algorithmic}[1]
\State $todo \gets \{\}$
\State $res \gets $ Liste vide
\State $s_0 \gets $ choisir un sommet arbitraire de $G$
\State $todo\gets G-\{s_0\}$
\State $res.\append(s_0)$
\State $s\gets s_0$
\While{$todo\not=\{\}$}
	\State $s' \gets$ le sommet le plus proche de $s$
	\State $todo\gets todo - \{s'\}$
	\State $res.\append(s')$
	\State $s\gets s'$
\EndWhile
\State\Return $res.\append(s_0)$
\end{algorithmic}
\end{algorithm}

%Complexité : $O(n^2)$ ???

