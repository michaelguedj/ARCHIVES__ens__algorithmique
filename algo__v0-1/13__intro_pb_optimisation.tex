

Nous considérons une veille radio que l’on trouve dans une brocante ; 
elle comporte une molette : $A$ ; et 3 boutons : $B$, $C$ et $D$.

\begin{itemize}
\item	$A$ permet de capter des fréquences (100 fréquences disponibles) ; 
\item	$B$, $C$ et $D$ sont trois boutons qui peuvent prendre 10 valeurs chacune 
	-- mais dont on ne connait la signification.
\end{itemize}

Une solution est la donnée d’un quadruplet $(a, b, c, d)$ 
où $a$, $b$, $c$ et $d$ indiquent des positions, 
de la molette
et des différents boutons.

Au total : $100 \times 10 \times 10 \times 10 = 100~000$ solutions possibles.

Le but est de trouver une station diffusant une chanson que l’on aime bien, et 
avec une bonne qualité de diffusion.

Pour chaque solution, i.e. chaque possibilité de quadruplet, on donne une note.


\subsubsection*{Exemple de notation.}

On entend :
\begin{itemize}
\item	Seulement des bruits de grésillement $\rightarrow 0/20$ ;
\item	Des voix, avec beaucoup de grésillement $\rightarrow 5/20$ ;
\item	Une chanson avec grésillement $\rightarrow 12/20$ ;
\item	Des voix ; avec bonne qualité d’écoute $\rightarrow 14/20$ ;
\item	Une chanson « sympathique » ; avec bonne qualité d’écoute $\rightarrow 17/20$ ;
\item	Une chanson que l’on aime davantage ; avec bonne qualité d’écoute $\rightarrow 19/20$.
\end{itemize}

Nous nous faisons face à un problème d’\textbf{optimisation} ; 
il s’agit, en effet, de trouver une solution qui \textbf{maximise} la note.

Etant donné le nombre de solutions possibles 
 ; on utilise alors une heuristique de résolution.
(Tester 100 000 configurations possibles n’est pas acceptable pour un humain).




