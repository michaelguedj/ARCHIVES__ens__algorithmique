


\subsection{Liste cha\^inée}

\begin{itemize}
\item[--] \textbf{est\_vide} : retourne $\True$ si la liste est vide ; et $\False$ sinon ;
\item[--] \textbf{taille} : retourne le nombre d'éléments dans la liste ;
\item[--] \textbf{insérer\_après} : insére un élément après l'élément passé en argument ;
\item[--] \textbf{insérer\_en\_tête} : insére un élément avant le premier élément de la liste ;
\item[--] \textbf{supprimer\_après} : insére un élément après l'élément passé en argument ;
\item[--] \textbf{supprimer\_en\_tête} : supprime le premier élément de la liste.
\end{itemize}



\subsection{Pile}

\begin{itemize}
\item[--] \textbf{est\_vide} : retourne $\True$ si la pile est vide et $\False$ sinon ;
\item[--] \textbf{taille} : renvoie le nombre d'éléments dans la pile ;
\item[--] \textbf{empiler} : ajoute un élément sur la pile ;
\item[--] \textbf{dépiler} : enlève un élément de la pile et le retourne ;
\item[--] \textbf{sommet} : retourne l'élément de tête sans le dépiler.
\end{itemize}



\subsection{File}
\begin{itemize}
\item[--] \textbf{est\_vide} : retourne $\True$ si la file est vide, et $\False$ sinon ;
\item[--] \textbf{taille} : retourne le nombre d'éléments dans la file ;
\item[--] \textbf{enfiler} : ajoute un élément dans la file ;
\item[--] \textbf{défiler} : retourne le prochain élément de la file, et le retire de la file.
\end{itemize}
