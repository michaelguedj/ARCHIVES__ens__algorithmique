




\begin{algorithm}[H]
\caption{\algo{dicho\_init}($t$ : tableau, $n$ : taille du tableau, $x$ : élément)}
\begin{algorithmic}[1]
\State $d \gets 0$ 
\State $f \gets n-1$ 
\State\Return $\algo{dicho}(t, d, f, x)$
\end{algorithmic}
\end{algorithm}

\begin{algorithm}[H]
\caption{\algo{dicho}($t$ : tableau, $d,f$ : indices, $x$ : élément)}
\label{dicho}
\begin{algorithmic}[1]
\State \textbf{if} $d > f$ \textbf{then} \Return $-1$ \textbf{end if}
	\green{\Comment{Non trouvé.}}

\If{$f=d$}
	\State \textbf{if} $t[d] = x$ \textbf{then} \Return $d$  
	\textbf{else} \Return $-1$ \textbf{end if} 
\EndIf

\State $m \gets  E(\frac{d+f}{2})$	\green{\Comment{Partie entière.}}

\If{$ t[m] = x $}
	\State\Return $m$
\EndIf

\If{$t[m] < x $}
	\State\Return $\algo{dicho}(t, m+1, f, x)$
\Else
	\State\Return $\algo{dicho}(t, d, m-1, x)$
\EndIf
\end{algorithmic}
\end{algorithm}


%%%%%%%%%%%%%%%%%%%%%%%%%%%%%%%%%%%%%%%%%%%%%%%%%%%%
%%%%%%%%%%%%%%%%%%%%%%%%%%%%%%%%%%%%%%%%%%%%%%%%%%%%
%%%%%%%%%%%%%%%%%%%%%%%%%%%%%%%%%%%%%%%%%%%%%%%%%%%%
\blue{
\begin{fTheorem}
La complexité de $\algo{dicho}$ est en $O(\log n)$.
\end{fTheorem}
}

\begin{proof}[Preuve (d'approximation)]
Soit $\mC(n)$ le nombre de comparaisons pour une instance de taille $n$.
On a, 
$$
\mC(n) = \gamma + \mC(\frac{n}{2}) 
	~~~~ ( \gamma \text{ constante} ) 
	$$
La deuxième instance appelée vérifie :
$$
\mC(\frac{n}{2}) = \gamma + \mC(\frac{n}{4})
	$$
D'où, 
$$
\mC(n) = \gamma + \mC(\frac{n}{2}) 
		= \gamma + \Big( \gamma + \mC(\frac{n}{4}) \Big)
		= 2\gamma + \mC(\frac{n}{4})
	$$
Soit, 
$$
\underline{\mC(n) = 2\gamma +  \mC(\frac{n}{2^2})
}
	$$
La troisième instance appelée vérifie :
$$
\mC(\frac{n}{2^2}) = \gamma +  \mC(\frac{n}{2^3})
	$$
D'où, 
$$
\underline{\mC(n) = 3\gamma +  \mC(\frac{n}{2^3})
}
	$$
En itérant, $\mC(n)$ s'écrit : 
$$
\underline{
\mC(n) = k\gamma +  \mC(\frac{n}{2^k})
}
	$$
Où $k$ correspond au $(k+1)$-ième appel récursif.

On a$^*$ :
$$
		\frac{n}{2^k} = 1 
	\Rightarrow
		\underline{ \mC(\frac{n}{2^k}) = \alpha \in\mathbb{N}}
	$$
Et : 
$$
		\frac{n}{2^k} = 1 
	\iff
		n = 2^k
	\iff 
		\underline{ \log_2 n = k}
	$$
$C(n)$ s'écrit alors :
$$
	\mC(n) = k\gamma +  \mC(\frac{n}{2^k})
	= 
		\underline{\log_2 (n) \gamma + \alpha \in O(\log n)}
	$$
\end{proof}

$(^*)$
Le cas $\frac{n}{2^k}<0$ 
(i.e. le cas d'une taille négative d'instance),
 prévu par l'algorithme
(ligne 1 de l'algorithme \ref{dicho}),
 est laissé en exercice.
Vous devriez trouver (si $n>0$) :
$$
\frac{n}{2^k}<0 \Rightarrow
 \frac{n}{2^{k-1}}=2
	\iff n = 2.2^{k-1} = 2^k
$$
(Considérer l'appel parent).