
%%%
\subsection{Définition du problème}
%%%
%%%
\blue{
\begin{fDefinition}[Problème de la somme du sous-ensemble]
$~$
\begin{itemize} 
	\item \textup{Entrée.}
		\begin{itemize}
			\item Un tableau $t$, de taille $n$, à valeurs entières.
			\item Une capacité $c\in \mathbb{N}$.
		\end{itemize}
	\item \textup{Problème.}
		\begin{itemize}
			\item[]Trouver $k$ indices de $t$,  distincts, soient $i_1, ..., i_k$, tels que la somme : 
			$$t[i_1] + ... + t[i_k]$$
			Approche la capacité $c$, sans la dépasser.
		\end{itemize}
\end{itemize}
\end{fDefinition}
}

\blue{
\begin{fTheorem}
Le problème de la somme du sous-ensemble
est NP-hard. 
\end{fTheorem}
}

\begin{proof}
Admis.
\end{proof}
%%%
%%%
%%%
%%%
%%%
%%%
\subsection{Heuristique gloutonne}
%%%
%%%

\begin{algorithm}[H]
\caption{$\algo{greedy}(t, n, c)$}
\begin{algorithmic}[1]
\State trier $t$ selon l'ordre décroissant
\State $res \gets $ Liste vide
\State $val \gets 0$
\For{$i \gets 0, ..., n-1$}
	\If{$val + t[i] \leq c$}
		\State $res.\append(i)$	
		\State $val \gets val + t[i]$
	\EndIf
\EndFor
\State\Return $res$
\end{algorithmic}
\end{algorithm}

Complexité : $O(n . \log n)$.

