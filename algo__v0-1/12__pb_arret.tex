

\blue{
\begin{fDefinition}[\textsc{Arrêt}]
$\\$
\noindent
\underline{Entrées :}
\begin{enumerate}
\item \textup{\texttt{<Prog>}} : le code source d'un programme \textup{\texttt{Prog}} ;
\item \textup{\texttt{x}} : une entrée pour \textup{\texttt{Prog}}.
\end{enumerate}
\underline{Sortie :} \textup{\texttt{Prog(x)}} s'arrête-t-il ?
\end{fDefinition}
}

\blue{
\begin{fTheorem}[Turing]
\textsc{Arrêt} est indécidable.
\end{fTheorem}
}
\begin{proof}[Preuve]
(Par l'absurde).
On suppose qu'\textsc{Arrêt} est décidable ; i.e.
il existe un programme, 
soit \texttt{Halt}, qui décide le problème de l'arrêt ;
i.e., pour tout programme \texttt{Prog} de code source \texttt{<Prog>}, 
pour toute entrée \texttt{x} de \texttt{Prog} : 
\begin{itemize}
\item
	\texttt{Prog(x)} s'arrête $\iff$
	\texttt{Halt(<Prog>, x)} répond \texttt{Vrai} ;
\item
	\texttt{Prog(x)} ne s'arrête pas $\iff$
	\texttt{Halt(<Prog>, x)} répond \texttt{Faux}.
\end{itemize}

On considère le programme \texttt{Diagonale} ci-après :

\begin{verbatim}
Diagonale(y):
    Si Halt(y, y) = Vrai :
        opérer une boucle infinie
    Sinon
        retourner "toto"
    Fin Si
\end{verbatim}

Nous considérons l'exécution : \texttt{Diagonale(<Diagonale>)}.
\begin{enumerate}[(a)]
\item \underline{Cas 1 : \texttt{Halt(<Diagonale>, <Diagonale>) = Vrai}}

D'après le code de \texttt{Diagonale}, il suit que 
\texttt{Diagonale(<Diagonale>)} ne s'arrête pas, donc que 
\texttt{Halt(<Diagonale>, <Diagonale>)} répond \texttt{Faux}.

\item \underline{Cas 2 : \texttt{Halt(<Diagonale>, <Diagonale>) = Faux}}

D'après le code de \texttt{Diagonale}, il suit que 
\texttt{Diagonale(<Diagonale>)} s'arrête, donc que 
\texttt{Halt(<Diagonale>, <Diagonale>)} répond \texttt{Vrai}.
\end{enumerate}
%
\end{proof}

%%%%%%%%%%%%%%%%%%%%%%%%%%%%%%%%%%%%%%%%%%%%%%%%%%%%%%%%%%%%%%%
%\Section{\textsc{Ramasse-miette}}
%\begin{definition*}[\textsc{Ramasse-miette}]
%$~$
%
%\noindent
%\underline{Entrées :}
%\begin{enumerate}
%\item \texttt{Prog} : un programme;
%\item \texttt{x} : une entrée pour \texttt{Prog}.
%\end{enumerate}
%\underline{Sortie :} \texttt{Prog(x)} s'arrête-t-il ?
%\end{definition*}
%
%\begin{theoreme*}[Turing]
%\textsc{Arrêt} est indécidable.
%\end{theoreme*}
%%
%\begin{proof}[Preuve]
%\end{proof}


