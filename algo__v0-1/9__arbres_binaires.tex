
\subsection{Définitions préliminaires}

\blue{
\begin{fDefinition}[Type \code{Noeud\_Binaire}]
Le type \textup{\code{Noeud\_Binaire}} est un triplet $(id, val, f_g, f_d)$ tel que :
\begin{enumerate}[(i)]
\item 
$id$ est un identifiant à valeur dans $\mathbb{N}\cup\{-1\}$ ;
\item 
$val\in V$ (valeur du noeud) ; (par exemple $V=\mathbb{R}$) ;
\item
$f_g$ (fils gauche) est de type \textup{\code{Noeud\_Binaire}} ;
\item
De même pour $f_d$ (fils droit).
\end{enumerate}
\end{fDefinition}
}

On considère le prédicat $Null$, définit sur \code{Noeud\_Binaire} par : 
$$
Null(n) \iff n.id = -1
$$

Soint un ensemble d'éléments de type \code{Noeud\_Binaire}, 
on pose :
$$parent(s):=\{s' : s'.f_g = s \text{ ou } s'.f_d = s\}$$

\blue{
\begin{fDefinition}[Arbre binaire]
Un arbre binaire $\mathcal{A}$ est un ensemble d'éléments de type
Noeud\_Binaire vérifiant : 
\begin{enumerate}[(i)]
\item (Existence d'une racine unique) 
$$\exists ! s\in\mathcal{A}, parent(s) = \emptyset$$
\item (Fils gauche et droit distincts) 
$$\forall s\in\mathcal{A}, s.f_g.id \not= s.f_d.id 
\text{  ou  } s.f_g.id = s.f_d.id = -1 
$$
\item (Parent unique)
$$\forall s\in\mathcal{A}, |parent(s)|\leq 1$$
\end{enumerate}
\end{fDefinition}
}

On pose :
$$A_{\mathcal{A}} := 
\{ (x,y) : x, y\in\mathcal{A}, 
x\in parent(y) \text{ ou } y\in parent(x)\}
$$
On pose $G_{\mathcal{A}}$ le graphe non orienté associé à l'arbre
binaire $\mathcal{A}$, défini par : $G_{\mathcal{A}} := (S, A)$, 
tel qu'il existe une bijections $\sigma: S \rightarrow \mathcal{A}$
 assurant :
$\forall x, y\in S$, 
$$
(x,y)\in A \Rightarrow (\sigma(x), \sigma(y)) \in A_{\mathcal{A}}
$$

\blue{
\begin{fTheorem}
La fonction $\alpha:  A \rightarrow A_{\mathcal{A}}$ définie par : 
$ (x, y) \mapsto (\sigma(x), \sigma(y))
$
est une bijection.
\end{fTheorem}
}

\begin{proof}[Preuve]
\begin{enumerate}
\item $\alpha$ est injective. 
$$\alpha(x,y) = \alpha(x', y') 
	$$
$$
\iff
(\sigma(x), \sigma(y)) = (\sigma(x'), \sigma(y'))
	$$
$$
\iff
\big(	\sigma(x) = \sigma(x') \text{ et } \sigma(y) = \sigma(y')  
			\big)
	$$
$$
\iff 
\big(	x = x' \text{ et } y = y'  
			\big)
$$
(en utilisant le fait que $\sigma$ est bijective donc en particulier injective).

\item $\alpha$ est surjective.
Soit $(x, y) \in A_{\mathcal{A}}$. 
$\big(\sigma^{-1}(x) , \sigma^{-1}(y )\big)$ 
est un antécédant de $(x, y)$.
\end{enumerate}
\end{proof}

\blue{
\begin{fDefinition}[Hauteur d'un noeud]
Soit $\mathcal{A}$ un arbre binaire, la hauteur d'un noeud $s\in\mathcal{A}$, 
notée $height(s)$, 
est définie par : $\forall s\in\mathcal{A}$,  
\begin{enumerate}[(i)]
\item $height(s) = 0$ si $s$ est racine de $\mathcal{A}$ ;
\item $height(s) = 0$ si $Null(s)$ ;
\item $height(s) = height(s') +1 $, où $parent(s) = \{s'\}$, sinon.
\end{enumerate}
\end{fDefinition}
}

\blue{
\begin{fTheorem}
Le graphe non orienté associé à un arbre binaire est 
connexe et sans cycle.
\end{fTheorem}
}

% \begin{proof}[Idée de preuve]
% Considérer le graphe orienté associé.
% Démontrer la connexité et l'acyclicité du graphe orienté associé ; 
% concernant l'acyclicité :
% la racine ne peut être dans le cycle orienté ;
% d'autre part un cycle implique l'existence de deux parents pour un noeud
% -- le noeud appartenant au cycle de hauteur minimum.
% En déduire la connexité et l'acyclicité du graphe non orienté associé
% \end{proof}

\blue{
\begin{fDefinition}[ABR]
Un arbre binaire de recherche (ABR) est soit un arbre vide ; soit un arbre binaire vérifiant,
pour  tout noeud $s$ : 
\begin{itemize}
	\item[--] $\forall s' \in G(s), s'.val \leq s.val $ ;
	\item[--] $\forall s' \in D(s), s.val < s'.val$ ;
\end{itemize}
où $G(s)$ (resp. $D(s)$) est le sous-arbre gauche (resp. droit) du noeud $s$.
\end{fDefinition}
}


%%%
\subsection{Recherche dans un ABR}

\begin{algorithm}[H]
\caption{\algo{recherche\_ABR} ($s\in\mathcal{A}$, $x\in V$ : valeur recherchée)}
\begin{algorithmic}[1]
\If{$Null(s)$}
	\State\Return $\False$
\ElsIf{$s.val = x$}
	\State\Return $\True$
\ElsIf{$s.val > x$}
	\State\Return $\algo{recherche\_ABR}(s.f_g, x)$
\Else
	\State\Return $\textsc{recherche\_ABR}(s.f_d, x)$
\EndIf
\end{algorithmic}
\end{algorithm}

Complexité : $O(\log n)$ en moyenne.

%%%
\subsection{Parcours infixe dans un arbre binaire}

\begin{algorithm}[H]
\caption{\algo{infixe} ($s\in\mathcal{A}$)}
\begin{algorithmic}[1]
\State \textbf{\textup{if}} $Null(s)$ 
	\textbf{\textup{then}} \textbf{\textup{ return}} \None \textbf{\textup{  end if}}
\If{$ \Not Null(s) $}
	\State $\algo{infixe}(s_g)$
\EndIf
\State $\print (s)$
\If{$\Not Null(s_d) $}
	\State $\algo{infixe}(s_d)$
\EndIf
\end{algorithmic}
\end{algorithm}

Complexité : $O(n)$.

\blue{
\begin{fTheorem}
Le parcours infixe d'un ABR donne une séquence des noeuds triés selon l'ordre croissant des valeurs.
\end{fTheorem}
}

\begin{proof}[Preuve] (Par récurrence sur la taille de l'ABR).
Si $|\mathcal{A}| = 1$, alors la proposition est vraie.

Supposons que, pour tout ABR de taille $m\leq k$, la proposition soit vraie.
Considérons un ABR de taille $k+1$ ; alors la séquence affichée est
de la forme :
\begin{center}
séquence affichée par $\algo{infixe}(s.f_g)$ ; $s$ ; 
séquence affichée par $\algo{infixe}(s.f_d)$. 
\end{center}
Par définition d'un ABR, 
\begin{itemize}
\item[--]  $\forall s' \in G(s), s'.val \leq s.val $ ;
\item[--] $\forall s' \in D(s), s.val < s'.val$.
\end{itemize}
D'autre part, l'hypothèse de récurrence nous assure que 
la séquence affichée par $\algo{infixe}(s.f_g)$ 
(resp. $\algo{infixe}(s.f_d)$) 
est conforme à
la proposition.
\end{proof}

