

\begin{algorithm}[H]
\caption{\algo{tri\_fusion}($lst$ : liste de taille $n$)}
\begin{algorithmic}[1]
\State\textbf{if} $n=1$
	\Return $lst$
\textbf{end if}
\State $m = E(n/2)$ \green{\Comment{Partie entière.}}
\State $lst_1 \gets \algo{tri\_fusion}(lst[0\rightarrow m-1])$
\State $lst_2 \gets \algo{tri\_fusion}(lst[m\rightarrow n-1])$
\State \Return $\algo{fusion}(lst_1, lst_2)$
\end{algorithmic}
\end{algorithm}

\begin{algorithm}[H]
\caption{\algo{fusion}($lst_1$ : liste de taille $n_1$, 
				      $lst_2$ : liste de taille $n_2$)}
\label{fusion}
\begin{algorithmic}[1]
\State $res\gets $ Liste vide
\While{non ($lst_1$ vide et $lst_2$ vide)}
	\If{$lst_1$ est vide}
		\State\Return $res + lst_2$
    \EndIf
	\If{$lst_2$ est vide}
		\State\Return $res + lst_1$
   \EndIf
	\If{$ \head(lst_1) \leq \head(lst_2)$}
		\State $res\gets res + [\head(lst_1)]$
		\State $lst_1 \gets \tail(lst_1)$
	\Else
		\State $res\gets res + [\head(lst_2)]$
		\State $lst_2 \gets \tail(lst_2)$	
	\EndIf
\EndWhile
\State\Return $res$
\end{algorithmic}
\end{algorithm}

\textbf{Exercice.}
La ligne 17 de l'algorithme \ref{fusion} n'est jamais atteinte.
Pourquoi ?

%%%%%%%%%%%%%%%%%%%%%%%%%%%%%%%%%%%%%%%%%%%%%%%%%%%%%%%%%
%%%%%%%%%%%%%%%%%%%%%%%%%%%%%%%%%%%%%%%%%%%%%%%%%%%%%%%%%
%%%%%%%%%%%%%%%%%%%%%%%%%%%%%%%%%%%%%%%%%%%%%%%%%%%%%%%%%
\blue{
\begin{fTheorem}
La complexité de $\algo{tri\_fusion}$ est en $O(n\log n)$.
\end{fTheorem}
}

\begin{proof}[Preuve (réduite aux cas des puissances de deux)]
%On pose $\mC(x)$ le nombre de comparaisons pour une liste
%de taille $x$.
Soit $p$ une puissance de $2$.

$$
\mC(p) = 1 + 2. \mC(\frac{p}{2}) + \gamma.p
$$
Où $\gamma$ est une constante.
On a de même :
$$
\mC(\frac{p}{2}) = 1 + 2. \mC(\frac{p}{4}) + \gamma.\frac{p}{2}
$$
Soit : 
$$
\mC(p) = 1 + 2 (1 + 2. \mC(\frac{p}{4}) + \gamma.\frac{p}{2}) + \gamma.p
$$
$$
\mC(p) = 1 + 2 + 4. \mC(\frac{p}{4}) + \gamma.p + \gamma.p
$$
$$
\mC(p) = 4. \mC(\frac{p}{4}) + 2\gamma.p + 3
$$
$$
\underline{\mC(p) = 2^2 . \mC(\frac{p}{2^2}) + 2\gamma.p + (2+1)}
$$
%
$$
\mC(\frac{p}{2^2}) = 1 + 2. \mC(\frac{p}{2^3}) + \gamma.\frac{p}{2^2}
$$
%
$$
\mC(p) = 2^2 . (1 + 2. \mC(\frac{p}{2^3}) + \gamma.\frac{p}{2^2}) + 2\gamma.p + (2+1)
$$
$$
\mC(p) =  2^2 + 2^2.2. \mC(\frac{p}{2^3}) + \gamma.p + 2\gamma.p + (2+1)
$$
$$
\underline{\mC(p) = 2^3. \mC(\frac{p}{2^3}) + 3\gamma.p 
	+ (2^2+2+1)}
$$
$$
\mC(\frac{p}{2^3}) =
	1 + 2. \mC(\frac{p}{2^4}) + \gamma. \frac{p}{2^3}
$$
$$
\mC(p) = 2^3. (1 + 2. \mC(\frac{p}{2^4}) + \gamma. \frac{p}{2^3}) + 3\gamma.p 
	+ (2^2+2+1)
$$
$$
\mC(p) = 2^3 + 2^3.2. \mC(\frac{p}{2^4}) + 2^3.\gamma. \frac{p}{2^3} + 3\gamma.p 
	+ (2^2+2+1)
$$
$$
\mC(p) = 2^3 + 2^4. \mC(\frac{p}{2^4}) + \gamma.p + 3\gamma.p 
	+ (2^2+2+1)
$$
$$
\underline{
\mC(p) = 2^4. \mC(\frac{p}{2^4}) + 4\gamma.p 
	+ (2^3+2^2+2+1)}
$$

En itérant, 
$$
\underline{
\mC(p) = 2^t. \mC(\frac{p}{2^t}) + t\gamma.p 
	+ (2^t+...+2^2+2+1)}
$$

On a :
$$
2^t+...+2^2+2+1 = 2^{t+1} - 1
$$

D'où, 
$$
\underline{
\mC(p) = 2^t. \mC(\frac{p}{2^t}) + t\gamma.p 
	+ 2^{t+1}-1}
$$

On a :
$$
\frac{p}{2^t} = 1 \iff p = 2^t \iff t = \log_2 p
$$
Et $\mC(1) = 1$ ; d'où :
$$
\mC(p) = 2^{\log_2 p}. 1 + \log_2( p) \gamma.p 
	+ 2^{\log_2 (p)+1}-1
$$
$$
\mC(p) = p + p.\log_2( p).\gamma 
	+ 2^{\log_2 (p)}.2+-1
$$
$$
\mC(p) = p + p.\log_2( p).\gamma 
	+ 2.p -1
$$
$$
\underline{\mC(p) = 
p.\log_2( p).\gamma+ 3.p -1} \in O(p\log p)
$$

\end{proof}

