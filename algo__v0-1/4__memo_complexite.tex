

\subsection{Approximation asymptotique}
\blue{
\begin{fDefinition}[Notation grand O]
Soient $f$ et $g$ deux fonctions de $\mathbb{N} \rightarrow \mathbb{R}^{+}$.
$f \in O(g)$ si :
\begin{itemize}
\item[--] $\exists K \in\mathbb{R}^{*+}$ ;
\item[--] $\exists n_0 \in \mathbb{N}$ ; 
\end{itemize}
et
$$
\forall n\geq n_0,~ f(n) \leq K. g(n)
$$
($f(n) \leq K. g(n)$ à partir d'un certain rang). 
\end{fDefinition}
}

%\noindent
\textbf{Exemples}
\begin{itemize}
\item[--] $7n-3 \in O(n)$
\item[--] $7n-3 \in O(n^2)$
\item[--] $987654321 + 10n^2 \in O(n^2)$
\end{itemize}

\noindent
\textbf{Remarque}
Le but est de trouver l'approximation la plus petite possible.


%%%%
%%%%
%%%%
%%%%
%%%%
\subsection{Complexités en temps classiques}
\begin{tabular}{|l|l|l|}
  \hline
  \textbf{Complexité} & \textbf{Notation asymptotique} & \textbf{Exemple} \\
  \hline
  Logarithmique 	& $O(\log n)$			& Recherche dichotomique dans un tableau trié. \\
  \hline
  Linéaire 		& $O(n)$ 				& Recherche séquentielle dans un tableau. \\
  \hline
  Quasi-linéaire  & $O(n \log n)$ 		& Tri fusion. \\
  \hline
  Quadratique 	& $O(n^2)$ 			& Tri sélection; tri à bulles.\\
  \hline
  Polynomiale 	& $O(n^k), k\geq 0$	& $~$... \\
  \hline
  Exponentielle 	& $O(k^n), k>1$ 		& Algorithme récursif pour Fibonacci. \\
  \hline
  Factorielle 	& $O(n!)$ 			& Résolution des $n$-reines par \textit{backtracking}. \\
  \hline
\end{tabular}

%%%%
%%%%
%\subsection{Comparaisons}
%$$
%\log n \ll n \ll n \log n \ll n^2 \ll n^3 \ll 2^n \ll n!
%$$
