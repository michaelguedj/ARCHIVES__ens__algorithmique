


%%%
%%%
\subsection{Afficher les éléments d'une matrice}
\begin{algorithm}[H]
\caption{\algo{afficher\_matrice} ($A$ : matrice $n\times m$)}
\begin{algorithmic}[1]
\For{$i \gets 0, ..., n-1$}	\green{\Comment{parcours sur les lignes}}
	\For{$j \gets 0, ..., m-1$}	\green{\Comment{parcours sur les colonnes}}
		\State \print($A_{i,j}$)
	\EndFor
	\State \print(saut de ligne)
\EndFor
\end{algorithmic}
\end{algorithm}

Complexité : $O(n\times m)$.

\noindent
Cas d'une matrice carré $n\times n$ : $O(n^2)$ (complexité linéaire !).

%%%
%%%
%%%
%%%
%%%
%%%
\subsection{Additionner deux matrices}
\begin{algorithm}[H]
\caption{\algo{additionner} ($A, B$ : matrice $n\times m$)}
\begin{algorithmic}[1]
\State $C \gets$ matrice $n\times m$
\For{$i \gets 0, ..., n-1$}
	\For{$j \gets 0, ..., m-1$}
		\State $C_{i, j}\gets A_{i,j} + B_{i,j}$
	\EndFor
\EndFor
\State\Return $C$
\end{algorithmic}
\end{algorithm}

Complexité : $O(n\times m)$.


%%%
%%%
%%%
%%%
%%%
%%%
\subsection{La matrice est-telle diagonale ?}

Rappel : la matrice carré $n\times n$, soit $M$, est diagonale si :
$$\forall i,j\in\{0, ..., n-1\}, 
i\not=j \Rightarrow M_{i,j}=0$$

\begin{algorithm}[H]
\caption{\algo{est\_diagonale} ($M$ : matrice $n\times n$)}
\begin{algorithmic}[1]
\For{$i \gets 0, ..., n-1$}
	\For{$j \gets 0, ..., n-1$}
		\If{$i\not=j$ et $M_{i,j}\not=0$}
			\State\Return $\False$
		\EndIf
	\EndFor
\EndFor
\State\Return $\True$
\end{algorithmic}
\end{algorithm}

Complexité : $O(n^2)$.

