


\blue{
\begin{fDefinition}[Heuristique]
Une heuristique (du grec ancien "eurisko" 
« je trouve ») est une méthode de calcul qui fournit "rapidement" une solution "réalisable" 
(non nécessairement optimale ou exacte)
 pour un problème d'optimisation "difficile".
\end{fDefinition}
}

\textbf{Note :} 
Une heuristique s'impose quand les algorithmes de résolution exacte 
sont de complexité exponentielle, et dans beaucoup de problèmes "difficiles". 
L'usage d'une heuristique est également pertinent pour calculer une solution approchée 
d'un problème, ou encore pour "accélérer" un processus de résolution exacte. 
Généralement, une heuristique est conçue pour un problème particulier, 
en s'appuyant sur sa structure propre, mais peut contenir des principes 
plus généraux.

\blue{
\begin{fDefinition}[Algorithme glouton]
Un algorithme glouton (greedy algorithm en anglais) est un algorithme qui 
suit le principe de faire, étape par étape, un choix optimum (local). 
Dans certains cas, cette approche permet d'arriver à un optimum global ; 
mais dans le cas général, c'est une heuristique.
\end{fDefinition}
}




